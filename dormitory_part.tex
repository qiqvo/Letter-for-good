\documentclass[14pt, a4paper]{extarticle}  % nsk: article -> extarticle
\usepackage[utf8]{inputenc}
\usepackage[T2A,T1]{fontenc}  % nsk

\evensidemargin0cm \oddsidemargin0cm \textwidth16cm
\textheight23cm \topmargin-2cm

% \usepackage[14pt]{extsizes}  % nsk
\usepackage[colorlinks = true,
            linkcolor = blue,
            urlcolor  = blue,
            citecolor = blue,
            unicode,  % nsk
            anchorcolor = blue]{hyperref}

\usepackage[english, ukrainian]{babel}

\usepackage{graphicx}
\usepackage{wrapfig}
\usepackage{lscape}
\usepackage{rotating}
\usepackage{epstopdf}
\usepackage{amsmath}
\usepackage{amsfonts}

\title{Колективний лист від студентів з приводу життя в гуртожитку}
% \author{Студенти ОМ-4}
\date{\today}
\begin{document}
\maketitle

Шановна адміністрація університету та гуртожитків, 

Більше ніж половина студентів щороку вступає до університету з інших міст, тому говорячи про проблеми освіти складно оминути проблеми, що пов'язані з гуртожитком. 
\begin{enumerate}
	\item Кожного року, в період майже з травня по вересень, в гуртожитку номер 16 немає гарячої води. Зазвичай її також немає і в частині Голосіївського району, але, якщо у квартирі можна встановити бойлер чи у будинку --- котел, то в гуртожитку цього досі не зробили. 
	\item Зважаючи, що гуртожиток знаходиться у 20 хвилинах пішки від корпусу, можна було б організувати пункт велосипедів у приміщенні бібліотеки, яка не працює. Влітку їх можна було б здавати в аренду не тільки студентам. 
	\item Дуже приємно, коли гуртожиток оновлюється. Але дуже неприємно жити в процесі оновлення, тобто у ремонті, серед постійного будівельного пилу. Останні 5 місяців саме це відбувається зі студентами, що живуть на верхніх поверхах у (``старому'') північному крилі. 
	\item Оновлення гуртожитків, може, краще починати з кімнат, аніж з коридорів. Наприклад, кімнати у (``старому'') північному крилі досі мають старі вікна.
	\item Робити оновлення треба якісно: після того, як влітку 2016 відремонтували блоки та встановили двері, виявилось, що частину дверей можна відкрити одним ключем. 
	\item Систему поселення студентів треба організувати справедливо, наприклад, як це зроблено в \href{https://dorms.online/}{ІМВ}. \href{https://docs.google.com/document/d/1Z2BHe0A4EmTV-15ywfvJOvfqsr-1LhsP0BfDYz7UGbw/edit}{Важлива записка} з їхнього сайту, що детальніше пояснює проблему. 
	\item Попри те, що корпус факультету Кібернетики знаходиться на ВДНГ, гуртожиток факультету знаходиться за межами студмістечка, у 20 хвилинах пішки. Якщо б він знаходився у студмістечку, то дістатися корпусу можна було б за 5 хвилин. Може є сенс зробити обмін гуртожитками з якимось з факультетів, що навчається у центрі міста. Просимо провести опитування студентів. 
\end{enumerate}

В будь-якому разі, окрім навчальних покращень важливо звернути не меншу увагу на проблеми проживання студентів.
\end{document}